%%----------LaTeX template for teachers-----------------------------%%
%%----------Samuel S. Watson----------------------------------------%% 
%%----------January 2013--------------------------------------------%% 
A Note will be attached here.
% http://math.mit.edu/~sswatson/latexforteachers.html

\documentclass[12pt]{article} % Specifies font size
%----------------PACKAGES-------------------------------------------%%
\usepackage[margin=1in]{geometry} % Sets all four margins to 1 inch
\usepackage[pdftex]{graphicx} % Allows inclusion of image files
\usepackage{amssymb} % Access to extra math symbols
\usepackage{amsmath} % Access to extra math symbols
\usepackage{wrapfig} % Allows wrapping of text around figures
\usepackage{calc} % Gives access to a basic calculator 
\usepackage{array} % Gives custom commands for array environments 
\usepackage{anyfontsize} % Lets you adjust the font size easily
%-------------------------------------------------------------------%%

%----------------COMMANDS-------------------------------------------%%
\newcommand\blank{\underline{\hspace{2cm}}} % Gives a blank 
\newcounter{prob} % A new counter for current problem number
\setcounter{prob}{1} % Start the counter at the value 1
\newcommand\itm{
\fbox{\textbf{\theprob}} \refstepcounter{prob}
} % Calls problem number
\newcommand{\problem}[1]{\makebox[0.5cm]{\itm}   
  \begin{minipage}[t]{\textwidth-0.5cm} #1 \end{minipage} 
} % An environment for a problem statement on or more lines
\newcommand{\pairofprobs}[2]{
  \begin{minipage}[t]{0.5\textwidth}\itm #1 \end{minipage} 
  \begin{minipage}[t]{0.5\textwidth}\itm #2 \end{minipage} 
} % Fits two problems on a line
\newcommand{\threeprobs}[3]{
\begin{minipage}[t]{0.31\textwidth}\itm #1 \end{minipage} \hfill
 \begin{minipage}[t]{0.31\textwidth}\itm #2 \end{minipage} \hfill 
 \begin{minipage}[t]{0.31\textwidth}\itm #3 \end{minipage}
} % Fits three problems on a line
\newcommand{\fourprobs}[4]{
\begin{minipage}[t]{0.21\textwidth}\itm #1 \end{minipage} \hfill
 \begin{minipage}[t]{0.21\textwidth}\itm #2 \end{minipage} \hfill 
 \begin{minipage}[t]{0.21\textwidth}\itm #3 \end{minipage} \hfill 
 \begin{minipage}[t]{0.21\textwidth}\itm #4 \end{minipage}
} % Fits four problems on a line


\newcounter{choice} % Counter for multiple choice problems 
\setcounter{choice}{1} % Start the counter at the value 1
\newcommand\achoice{
(\alph{choice}) \stepcounter{choice}
} % Generates letter for multiple choice option
\newcommand{\answers}[5]{\vspace*{-7mm} 
  \begin{tabular}{l@{\hspace{1mm}}p{0.9\textwidth}}
    \achoice & #1 \\ \achoice & #2 \\ \achoice & #3 \\ 
    \achoice & #4 \\ \achoice & #5 \end{tabular}
  \setcounter{choice}{1}
} % Makes multiple-choice options 
%---------------------------------

% The commands below are for setting up arithmetic 
% problems with the four basic operations. See examples 
% in the CONTENT section  

\newcommand\divi[2]{
#1 \: \begin{array}{|l}
\hline #2
\end{array}
}

\newcommand\mult[2]{
$\begin{array}{rr} 
 & #1 \\ 
 \times & #2 \\ \hline 
 \end{array}$}
 
\newcommand\addi[2]{
  $\begin{array}{rr} 
   &  #1 \\ 
    + & #2 \\ \hline 
  \end{array}$}

\newcommand\subt[2]{
  $\begin{array}{rr}
    & #1 \\ 
    - & #2 \\ \hline
  \end{array}$}
%-------------------------------------------------------------------%%

%-----------FORMATTING----------------------------------------------%%
\pagestyle{empty} % Ensures that no page numbers are printed
\parskip = 0.2 in % Puts a little space between paragraphs 
\parindent = 0.0 in % Enforces no indentation for paragraphs
%-------------------------------------------------------------------%%

%-----------USAGE EXAMPLES------------------------------------------%%
% To try any of the examples below, uncomment them and paste 
% them below the \begin{document} command in the CONTENT section. 

% To set up a division problem such as 93 divided by 3:
% \divi{3}{93}

% To set up a muliplication problem such as 14 times 4:
% \mult{14}{4}

% To put two problems on the same line: 
% \pairofprobs{\divi{3}{93}}{\mult{14}{4}} 

% To include a 3cm vertical space between questions 
% \vspace{3cm} 

%--------------------------------------------------------------------%%

%-----------CONTENT--------------------------------------------------%%
\begin{document}

{\fontsize{14}{17}\selectfont
%%%%%%%%%%%%%%%%%% Part 1
\begin{center}
  \textsc{ Completing the Square}
\end{center}

{$\textbf{Set A}$.} 
Expand these squares of binomials.

\threeprobs
{$(x+2)^2 $}
{$(x-5)^2 $}
{$(2x+1)^2$}

\threeprobs
{$(a+b)^2$}
{$(a-b)^2$}
{$(2a+2b)^2$}

\setcounter{prob}{1}

{$\textbf{Set B}$.}  Expand these products of binomials.

\threeprobs
{$(x+1)(x+2)$}
{$(2x+1)(x+1)$}
{$(x-3)(x+3)$}

\setcounter{prob}{1}

{$\textbf{Set C}$.} Complete the square of each expression.

\threeprobs
{$x^2+2x$}
{$x^2+4x$}
{$x^2-10x$}

\threeprobs
{$3x^2+6x$}
{$2x^2+4x$}
{$5x^2-10x$}

\setcounter{prob}{1}

{$\textbf{Set D}$.} Complete the square of each expression.

\threeprobs
{$x^2+2x+1$}
{$x^2+4x-2$}
{$x^2-10x+2$}

\threeprobs
{$3x^2+6x-3$}
{$2x^2+4x+1$}
{$5x^2-10x-2$}

\setcounter{prob}{1}

{$\textbf{Set E}$.} Vertex form of a Parabola

\noindent
{The equation of a parabola can be written as $y=a(x-h)^2 +k$. The values $h$ and $k$ give us the vertex $(h,k)$ of the parabola. Complete the square of each to find its vertex.}

\threeprobs
{$y=x^2+2x+1$}
{$y=x^2+4x-2$}
{$y-x^2-10x+2$}

\setcounter{prob}{1}
{$\textbf{Set F}$.} Standard form of a Circle

\noindent
{The equation of a circle can be written as $(x-h)^2 + (y-k)^2 = r^2$. The values $h$ and $k$ give us the center $(h,k)$ of the circle, and $r$ is the radius. Complete the square of each to find its center and radius.}

\pairofprobs
{$x^2+y^2+6x-4y=12$}
{$x^2+y^2+4x-6y+9=0$}


%\setcounter{prob}{1}
%{$\textbf{Set F}$.} Factor each expression %fully.

%\threeprobs
%{$x^2 + 2x + 1$}
%{$9x^2-4$}
%{$x^2 +5x +6$}

} % end font size setting


\end{document}
%--------------------------------------------------------------------%%